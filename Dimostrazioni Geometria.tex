\documentclass[a4paper]{report}
	
	\usepackage[italian]{babel}
	\usepackage[utf8]{inputenc}
	\usepackage{graphicx}
	\usepackage{enumitem}
	\usepackage{amsmath}
	\usepackage{latexsym}
	\usepackage{physics}
	\usepackage{wasysym}
	\usepackage{geometry}
	\geometry{a4paper, top=2cm, bottom=2cm, left=2cm, right=2cm, heightrounded}
	
	\title{\textbf{Dimostrazioni Interessanti di Geometria}}
	\author{R.Senese}
	\date{gennaio 2017}
	
	\begin{document}
		
		\maketitle
		%\tableofcontents
		\chapter{Insiemistica}
		\section{\textbf{Formula di Grassmann}}
			\[ dim(V+U) = dim(V) + dim(U) - dim(V \cap U) \] 
			\emph{Dimostrazione:} sia ${\{w_1, \dotsc w_n\}} $ una base di ${(V \cap U)}$. Possiamo estenderla a una base di ${V=\{w_1, \ldots, w_k, v_{k+1}, \ldots, v_n\}}$ e a una base di ${U= \{ w_1, \ldots, w_k, u_{k+1}, \ldots, u_m\}}$. Verifichiamo che \[\Big\{w_1, \ldots, w_k, v_{k+1}, \ldots, v_n, u_{k+1}, \ldots, u_m\Big\} \] rappresenta una base di ${(V+U)}$.
			Per dimostrare che è una base dimostriamo che è indipendente e che genera:
			\begin{itemize}
			\item \textbf{Indipendenza:} 
				\[
				\alpha_1w_1+\ldots+\alpha_kw_k+\beta_{k+1}v_{k+1}+\ldots+\beta_nv_n+\gamma_{k+1}u_{k+1}+\ldots+\gamma_mu_m=0 \quad \Rightarrow
				\]
				\begin{equation}\label{eq:indipendenzagrassmann}
				\Rightarrow \alpha_1w_1+\ldots+\alpha_kw_k+\gamma_{k+1}v_{k+1}+\ldots+\gamma_nv_n=-(\beta_{k+1}u_{k+1}+\ldots+\beta_mu_m)
				\end{equation}
				Poiché il primo mebro della equazione \ref{eq:indipendenzagrassmann} ${\in V}$ e il secondo ${\in U}$, allora per il segno di uguaglianza sarà:
				
				\[ -(\beta_{k+1}u_{k+1}+\ldots+\beta_mu_m) \in V\cap U \Rightarrow \]
				\[ \Rightarrow\exists \; t_1w_1+\ldots+t_kw_k=-(\beta_{k+1}u_{k+1}+\ldots+\beta_mu_m)\]
				ma poiché per ipotesi i vettori erano \emph{linearmente indipendenti} si ha:
				\[ t_1=\ldots=t_k=\beta_{k+1}=\ldots=\beta_m=0\]
			\item\textbf{generano:}
				\begin{equation*}\begin{split}
				\forall a\in(V+U)\ t.c.\ a=v+u\ per \ qualche \ v\in V e\ u\in U. \ Poich\acute{e}\ v\in \big\langle w_1, \ldots, w_k, v_{k+1} \ldots, v_n \big\rangle \\ e \ u\in\big\langle w_1, \ldots, w_k, u_{k+1}, \ldots, u_m \big\rangle, allora\ anche\ a\in\big\langle w_1, \ldots, w_k, v_{k+1}, \ldots, v_n, u_{k+1}, \ldots, u_m \big\rangle
				\end{split}\end{equation*}
			\end{itemize}
			\begin{flushright}
			\Square
			\end{flushright}
			
			
	\end{document}